% No 'submit' option for the problems by themselves.
\documentclass{harvardml}
% Use the 'submit' option when you submit your solutions.
%\documentclass[submit]{harvardml}

% Put in your full name and email address.
\name{Luis Antonio Perez}
\email{luisperez@college.harvard.edu}

% List any people you worked with.
\collaborators{%
}

% You don't need to change these.
\course{CS281-F15}
\assignment{Assignment \#0}
\duedate{11:59pm September 11, 2013}

\usepackage{url, enumitem}
\usepackage{amsfonts, amsmath}
\usepackage{listings}

% Some useful macros.
\newcommand{\given}{\,|\,}
\newcommand{\R}{\mathbb{R}}
\newcommand{\E}{\mathbb{E}}
\newcommand{\var}{\text{var}}
\newcommand{\cov}{\text{cov}}

\begin{document}

%%%%%%%%%%%%%%%%%%%%%%%%%%%%%%%%%%%%%%%%%%%%%%%%%%%%%%%%%%%%%%%%%%%%%%%%%%%%%%%%
\noindent Please turn in your solutions \textbf{via canvas} by the due date listed above.
\\

\noindent This assignment is intended to ensure that you have the background required for CS281. You should be able to answer the problems below without complicated calculations. All questions are worth $70/6 = 11.\bar{6}$ points unless stated otherwise.

\begin{problem}[Variance and covariance]
Let $X$ and~$Y$ be two independent random variables.

\begin{enumerate}[label=(\alph*)]
\item Show that the independence of~$X$ and~$Y$ implies that their
covariance is zero.

\item For a scalar constant~$a$, show the following two properties:
\begin{align*}
  \E(X + aY) &= \E(X) + a\E(Y)\\
  \var(X + aY) &= \var(X) + a^2\var(Y)
\end{align*}
\end{enumerate}
\end{problem}

% You can write your solution here.

%%%%%%%%%%%%%%%%%%%%%%%%%%%%%%%%%%%%%%%%%%%%%%%%%%%%%%%%%%%%%%%%%%%%%%%%%%%%%%%%

\begin{problem}[Densities]
Answer the following questions.
\begin{enumerate}[label=(\alph*)]
  \item Can a probability density function (pdf) ever take values greater than 1?
  \item Let $X$ be a univariate normally distributed random variable with mean 0 and variance $1/100$. What is the pdf of $X$?
  \item What is the value of this pdf at 0?
  \item What is the probability that $X = 0$?
  \item Explain the discrepancy.
\end{enumerate}
\end{problem}

%%%%%%%%%%%%%%%%%%%%%%%%%%%%%%%%%%%%%%%%%%%%%%%%%%%%%%%%%%%%%%%%%%%%%%%%%%%%%%%%

\begin{problem}[Conditioning and Bayes' rule]
Let $\mu \in \R^m$ and $\Sigma, \Sigma' \in \R^{m \times m}$. Let $X$ be an $m$-dimensional random vector with $X \sim \mathcal{N}(\mu, \Sigma)$, and let $Y$ be a $m$-dimensional random vector such that $Y \given X \sim \mathcal{N}(X, \Sigma')$. State how each of the following is distributed. (For example: ``normally distributed with mean $z$ and variance $s$.")
\begin{enumerate}[label=(\alph*)]
  \item The unconditional distribution of $Y$.
  \item The joint distribution of the random variable $(X,Y)$.
  \item The conditional distribution of $X$ given $Y$.
\end{enumerate}
\end{problem}

%%%%%%%%%%%%%%%%%%%%%%%%%%%%%%%%%%%%%%%%%%%%%%%%%%%%%%%%%%%%%%%%%%%%%%%%%%%%%%%%

\begin{problem}[I can Ei-gen]
Let $X \in \R^{n \times m}$.
\begin{enumerate}[label=(\alph*)]
\item What is the relationship between the $n$ eigenvalues of $XX^T$ and the $m$ eigenvalues of $X^TX$?
\item Suppose $X$ is square (i.e., $n=m$) and symmetric. What does this tell you about the eigenvalues of $X$? What are the eigenvalues of $X + I$, where $I$ is the identity matrix?
\item Suppose $X$ is square, symmetric, and invertible. What are the eigenvalues of $X^{-1}$?
\end{enumerate}
\end{problem}

%%%%%%%%%%%%%%%%%%%%%%%%%%%%%%%%%%%%%%%%%%%%%%%%%%%%%%%%%%%%%%%%%%%%%%%%%%%%%%%%

\begin{problem}[Calculus]
Let $x, y \in \R^m$ and $A \in \R^{m \times m}$. Please answer the following questions, writing your answers in vector notation.
\begin{enumerate}[label=(\alph*)]
\item What is the gradient with respect to $x$ of $x^T y$?
\item What is the gradient with respect to $x$ of $x^T x$?
\item What is the gradient with respect to $x$ of $x^T A x$?
\item What is the gradient with respect to $x$ of $A x$?
\end{enumerate}
\end{problem}

%%%%%%%%%%%%%%%%%%%%%%%%%%%%%%%%%%%%%%%%%%%%%%%%%%%%%%%%%%%%%%%%%%%%%%%%%%%%%%%%

\begin{problem}[Sanity check. This problem is worth 0 points.]
Bugs in machine learning software, as in any software, can be detected by running
tests to see if the code produces the required output for a particular input.
When one of these tests fails it is necessary to fix the code.  How do you usually track down where
the bug occurs?
\begin{itemize}
\item[A)] I use the debugger to figure out why things are behaving the way they are.
\item[B)] I add print statements to see the value of the different variables.
Scrolling through the resulting output is easier and more efficient than using the debugger.
\item[C)] If you are good at programming your code should not have bugs.
\end{itemize}

\end{problem}

%%%%%%%%%%%%%%%%%%%%%%%%%%%%%%%%%%%%%%%%%%%%%%%%%%%%%%%%%%%%%%%%%%%%%%%%%%%%%%%%

\begin{problem}[Stability]
One way to solve numerical problems in the evaluation of a function is to compute
a power series approximation around the input that causes the problems.
Around what input is the following python function not numerically stable?
What is the problem?

\begin{lstlisting}[language=python]
import numpy as np
def buggy(x):
    return np.sin(x**2) / x
\end{lstlisting}
Use the power series approximation to write a numerically stable linear approximation of the function around the the problematic input. (Hint: recall L'H\^opital's rule.) Your answer should be of the following form.

\begin{lstlisting}[language=python]
import numpy as np
def linear_approximation_around_problematic_input(x):
    return ?? # return some expression here
\end{lstlisting}
\end{problem}

\end{document}
